\documentclass[11pt,british,a4paper]{article}
%\pdfobjcompresslevel=0
\usepackage[usenames,dvipsnames]{xcolor}
\usepackage[includeheadfoot,margin=0.8 in]{geometry}
\usepackage{siunitx,physics,cancel,upgreek,varioref,listings,booktabs,pdfpages,ifthen,polynom,todonotes}
%\usepackage{minted}
\usepackage[backend=biber]{biblatex}
\DefineBibliographyStrings{english}{%
      bibliography = {References},
}
\addbibresource{sources.bib}
\usepackage{mathtools,upgreek,bigints}
\usepackage{babel}
\usepackage{graphicx}
\usepackage{float}
\usepackage{amsmath}
\usepackage{amssymb}
\usepackage{tocloft}
\usepackage[T1]{fontenc}
%\usepackage{fouriernc}
% \usepackage[T1]{fontenc}
\usepackage{mathpazo}
% \usepackage{inconsolata}
%\usepackage{eulervm}
%\usepackage{cmbright}
%\usepackage{fontspec}
%\usepackage{unicode-math}
%\setmainfont{Tex Gyre Pagella}
%\setmathfont{Tex Gyre Pagella Math}
%\setmonofont{Tex Gyre Cursor}
%\renewcommand*\ttdefault{txtt}
\graphicspath{{figs/}}
\usepackage[scaled]{beramono}
\usepackage{fancyhdr}
\usepackage[utf8]{inputenc}
\usepackage{textcomp}
\usepackage{lastpage}
\usepackage{microtype}
\usepackage[font=normalsize]{subcaption}
\usepackage{luacode}
\usepackage[linktoc=all, bookmarks=true, pdfauthor={Anders Johansson},pdftitle={FYS-STK4155 Project 1}]{hyperref}
\usepackage{tikz,pgfplots,pgfplotstable}
\usepgfplotslibrary{colorbrewer}
\usepgfplotslibrary{external}
\tikzset{external/system call={lualatex \tikzexternalcheckshellescape -halt-on-error -interaction=batchmode -jobname "\image" "\texsource"}}
\tikzexternalize%[prefix=tmp/, mode=list and make]
\pgfplotsset{cycle list/Dark2}
\pgfplotsset{compat=1.8}
\renewcommand{\CancelColor}{\color{red}}
\let\oldexp=\exp
\renewcommand{\exp}[1]{\mathrm{e}^{#1}}


\labelformat{section}{#1}
\labelformat{subsection}{exercise~#1}
\labelformat{subsubsection}{paragraph~#1}
\labelformat{equation}{equation~(#1)}
\labelformat{figure}{figure~#1}
\labelformat{table}{table~#1}

\renewcommand{\footrulewidth}{\headrulewidth}

%\setcounter{secnumdepth}{4}
\setlength{\parindent}{0cm}
\setlength{\parskip}{1em}

\definecolor{bluekeywords}{rgb}{0.13,0.13,1}
\definecolor{greencomments}{rgb}{0,0.5,0}
\definecolor{redstrings}{rgb}{0.9,0,0}
\lstset{rangeprefix=!/,
    rangesuffix=/!,
    includerangemarker=false}
\lstset{showstringspaces=false,
    basicstyle=\small\ttfamily,
    keywordstyle=\color{bluekeywords},
    commentstyle=\color{greencomments},
    numberstyle=\color{bluekeywords},
    stringstyle=\color{redstrings},
    breaklines=true,
    %texcl=true,
    language=Fortran,
    morekeywords={norm2,class,deferred}
}
\colorlet{DarkGrey}{white!20!black}
\newcommand{\eqtag}[1]{\refstepcounter{equation}\tag{\theequation}\label{#1}}
\hypersetup{hidelinks=True}

\sisetup{detect-all}
\sisetup{exponent-product = \cdot, output-product = \cdot,per-mode=symbol}
% \sisetup{output-decimal-marker={,}}
\sisetup{round-mode = off, round-precision=3}
\sisetup{number-unit-product = \ }

\allowdisplaybreaks[4]
\fancyhf{}

\rhead{Project 2}
\rfoot{Page~\thepage{} of~\pageref{LastPage}}
\lhead{FYS-STK4155}

%\definecolor{gronn}{rgb}{0.29, 0.33, 0.13}
\definecolor{gronn}{rgb}{0, 0.5, 0}

\newcommand{\husk}[2]{\tikz[baseline,remember picture,inner sep=0pt,outer sep=0pt]{\node[anchor=base] (#1) {\(#2\)};}}
\newcommand{\artanh}[1]{\operatorname{artanh}{\qty(#1)}}
\newcommand{\matrise}[1]{\begin{pmatrix}#1\end{pmatrix}}
\DeclareMathOperator{\Proj}{Proj}
\DeclareMathOperator{\Col}{Col}
\DeclareMathOperator{\sgn}{sgn}
\DeclareMathOperator{\MSE}{MSE}

\newread\infile

\setcounter{tocdepth}{2}
\numberwithin{equation}{section}

\pgfplotstableset{
    every head row/.style={before row=\toprule,after row=\midrule},
    every last row/.style={after row=\bottomrule},
    columns/Method/.style={string type, column type=l}
}

%start
\begin{document}
\title{FYS-STK4155: Project 2}
\author{Anders Johansson}
%\maketitle

\begin{titlepage}
%\includegraphics[width=\textwidth]{fysisk.pdf}
\vspace*{\fill}
\begin{center}
\textsf{
    \Huge \textbf{Project 2}\\\vspace{0.5cm}
    \Large \textbf{FYS-STK4155 --- Applied data analysis and machine learning}\\
    \vspace{8cm}
    Anders Johansson\\
    \today\\
}
\vspace{1.5cm}
\includegraphics{uio.pdf}\\
\vspace*{\fill}
\end{center}
\end{titlepage}
\null
\pagestyle{empty}
\newpage

\pagestyle{fancy}
\setcounter{page}{1}

\begin{abstract}
    This project applies several statistical learning methods to data from the famous Ising model.
    Linear regression is used to determine the coupling, while classification of spin configurations as ordered or disordered is achived with both logistic regression and multi-layered naural networks.
    Finally, neural networks are trained to predict the energy of a spin configuration.
    Ridge and LASSO regression are found to outperform ordinary least squares when applied to test data, with LASSO being able to break the symmetry of the coupling and give the correct coupling matrix.
\end{abstract}

\tableofcontents

\section{Introduction}
The Ising model is famous in many fields of science.
Physicists study the model because it represents a magnetic lattice and can model ferromagnetic materials, and also because it has a large university class and has the same behaviour as many other models.
Social scientists like the model because it is a simple correlated system, which can approximate correlated quantities such as dialects.
The Metropolis algorithm, acknowledged as one of the premiere algorithms in computational science, is a method for finding the properties of the Ising model numerically.

This project does not deal with the simulation of a spin lattice, but rather the classification and analysis of precalculated spin configurations with corresponding energies.
Thus the Ising model can be used to verify various numerical methods for both regression and classification, and also measure their performance to infer their relative strengths and weaknesses.

The main goal is to compare simple neural networks with the different linear regression methods for regression and logistic regression for classification.
The regression part of the problem is to find the coupling constants from given spin configurations and their energies, as well as prediction of energies.
Binary classification of states as ordered or disordered is used as a classification problem.

Consequently, the first part of this report derives the necessary mathemetical tools for both regression and classification, including the back propagation algorithm for training neural networks. Different methods for classification and regression are then applied to the Ising data provided in\cite{mehta} and compared.




\clearpage
\nocite{*}
\printbibliography{}
\addcontentsline{toc}{section}{\bibname}
\end{document}
